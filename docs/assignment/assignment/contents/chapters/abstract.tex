

\begin{abstract}

\begin{multicols}{2}
Das vorliegende Dokument beschreibt das Produkt SmartCart des Teams
IoT-Designers. Das Ziel der Anwendung ist es einen smarter Way of
Shopping zu erm�glichen. The application offers the possibility of easily
marking items as added to the cart and the navigating through a list via
gestures, um den User beim Eimkaufen zu unterst�zten und entlasten. Therefore,
the recognition of gestures is done via the built-in acceleration sensor. 

Anstatt umst�ndlich mit Stift und Papier die Einkaufsliste abzuarbeiten soll
diese App durch die Gesternsteuerung erleichertern. Dabei fokusiert sich diese
App auf zwei use cases welche dem Benutzer angeboten werden. Dem Benutzer soll
es erm�glicht werden mit einer Hand und einfachen jedoch eindeutigen Gesten
sowohl zwischen den Items in der Einkaufsliste hin und her zu wechseln als auch
die bereits gefundenen Items abzuhaken.

To recognize the chosen gestures, the acceleration sensor and the gyroscope are
used. The retrieved acceleration values are used to determine the movement that
is made. The gyroscope sensor serves to recognize the orientation of the
smartphone and to be able to subtract out its influence out of the acceleration
values.

Um aus den von den Sensoren erhaltenen Werten Gesten abzuleiten wurden zun�chst
Daten erfasst, welche die jeweilige Gesten wiedergeben. Damit die verschiedenen
Sensorwerte zuordbar sind wurden viele Messungen durchgef�hrt. Damit die Gesten
m�glichst unabh�ngig der Ausrichtung des Mobile Phones durchf�hrbar sind
werden verschiedene mathematischen Berechnungen ben�tigt.

Das Ende dieses Dokuments beschreibt das Ergebnis der App sowie deren Nutzung.

\end{multicols}
\end{abstract}