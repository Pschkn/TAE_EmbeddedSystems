In the following sections, the idea and implementation of Smart Cart -- an
Android application that will simplify your shopping experience -- will be
exposed. The application was created in the course of the workshop
``\assignmentName'' by the team IoT-Designers:

\begin{table}[h]
\renewcommand\arraystretch{1}
\centering
\begin{tabular}{p{0.21\textwidth}p{0.21\textwidth}p{0.21\textwidth}p{0.21\textwidth}}
\begin{center}Timo\end{center} & 
\begin{center}Wojciech\end{center} & 
\begin{center}Markus\end{center} & 
\begin{center}Simon\end{center}
\\
\vspace{-1.25cm}\begin{center}Acquistapace\end{center} & 
\vspace{-1.25cm}\begin{center}Lesnianski\end{center} & 
\vspace{-1.25cm}\begin{center}Just\end{center} & 
\vspace{-1.25cm}\begin{center}Schneider\end{center}
\\
\vspace{-1cm}\begin{center}\includegraphics[width=0.2\textwidth]{res/intro/Timo.png}\end{center}
& 
\vspace{-1cm}\begin{center}\includegraphics[width=0.2\textwidth]{res/intro/Ich.png}\end{center}
&
\vspace{-1cm}\begin{center}\includegraphics[width=0.2\textwidth]{res/intro/Markus.png}\end{center}
&
\vspace{-1cm}\begin{center}\includegraphics[width=0.2\textwidth]{res/intro/Simon.png}\end{center}
\\
\vspace{-1cm}\begin{center}Project Leader\end{center} & 
\vspace{-1cm}\begin{center}Developer\end{center} & 
\vspace{-1cm}\begin{center}Data Analyst\end{center} &
\vspace{-1cm}\begin{center}Documentation Manager \end{center}
\end{tabular}
\end{table}

\section{Introduction}
Since the idea of SmartCart evolved during the workshop, this section firstly
introduces the initial product idea of Smart Cart and the change of scope that
project went through. Later on, the architecture, the state machine and the
technology stack that is used are described. For the purpose of developing the
application Smart Cart, a data collection and data analysis had to take place.
These steps are described in the section \ref{sect:dataAnalysis}.

\subsection{Initial Idea of Smart Cart}
The first concept of smart cart was to offer its user the possibility to add
items to a shopping list and to get this shopping list ordered automatically as
the user enters a supermarket. The entered grocery is determined with the help
of the Here API. Based on the knowledge of the accessed shop and the ordering of
its departments, the items that were previously added to the shopping list,
should be ordered.

\subsection{Change of Scope and final Idea of Smart Cart}
Even though the initial idea of Smart Cart would have been a very helpful
application to the user, it is strongly based on the collaboration with the
operators of the supported supermarkets. This is especially true for the data
acquisition regarding the offered products and the available departments of a
supermarket. Therefore, the initial scope of the application was changed towards
an application that is less dependent on master data.

The revised concept of Smart Cart focusses more on the interaction of the user
and the application. It omits the features of recognising a shop that is entered
and ordering the list of shopping items according to the recognised shop.
Instead, the application should offer the possibility of easily marking an item
as �added to the cart� and of navigating through the list via gestures. The
recognition of gestures is done via the built-in sensors for acceleration and
the gyroscope.

Even the initial idea was discarded during the workshop for the mentioned
reasons, the focus that is now put on the user interaction might might also
support the initial idea. 
