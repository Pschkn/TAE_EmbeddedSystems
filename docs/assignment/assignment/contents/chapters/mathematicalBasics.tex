\subsubsection{Acceleration depending on Pitch}

\begin{figure}[htb]
    \centering
    \begin{minipage}{0.5\textwidth}
        \centering
        \captionsetup{justification=centering}
        
  \begin{tikzpicture}
	\newcommand{\cosAngle}{0.866}
	\newcommand{\sinAngle}{0.5}
	\newcommand{\sideLength}{2}
	\newcommand{\mobileLength}{4}
	\newcommand{\hlh}{\mobileLength/2}
	\newcommand{\mobileThickness}{0.25}
    	
	\coordinate (origo) at (0,0);
    	\coordinate (point_on_x) at (2,0);

	\coordinate (end_a_y) at (-\hlh - \sideLength*\cosAngle,-1);
	

   	\begin{scope}[rotate=30]
		\coordinate (end_a_y_mobile) at (-\hlh-\cosAngle*\sideLength,0);
		\coordinate (begin_a_y) at (-\hlh, 0);

    		\fill[gray] ($ (origo) + (-\hlh,-\mobileThickness/2) $) rectangle ($ (origo) + (\hlh,\mobileThickness/2) $);
		\draw[dashed, black] (-\hlh,0)--(\hlh,0);
		\coordinate (rotated) at (\hlh,0);
		 \draw[thick,orange,->] (begin_a_y)  -- (end_a_y_mobile) node (y_mobile)
		 [orange,below] {$a_{y, mobile}$}; \begin{scope}[rotate=-30]
			\draw[very thick,red,->] (-\hlh*\cosAngle,-1)  -- (end_a_y) node (y_real)
			[red,above] {$a_y$}; 
			\draw[thick,blue,->](end_a_y_mobile)
						--
						(-\hlh - \sideLength*\cosAngle,-1) node (z_mobile) [blue, left] {$a_{z,
						mobile}$};
		\end{scope}
	\end{scope}	
    	% draw axes
    	\fill[black] (origo) circle (0.05);
    	\draw[thick,orange,->] (origo) -- ++(2,0) node[black,right] {$y$};
    	\draw[thick,blue,->] (origo) -- ++(0,-2) node (mary) [black,below] {$z$};
   	\draw[thick] ($ (origo) + (-1,0) $) -- ($ (origo) + (1,0) $);

	% angle axes
	\tkzMarkAngle[fill= orange,size=1.2cm,opacity=.4](point_on_x,origo,rotated)
    	\pic [draw, ->, "$\alpha$", angle eccentricity=1.5] {angle =
    	point_on_x--origo--rotated};

	% angle vectors
	\tkzMarkAngle[fill=
	orange,size=1.2cm,opacity=.4](end_a_y,begin_a_y,end_a_y_mobile) \pic [draw, ->,
	"$\alpha$", angle eccentricity=1.5] {angle = end_a_y--begin_a_y--end_a_y_mobile};
  \end{tikzpicture}
        \caption{$a_y$ depending on Pitch}
    \end{minipage}% <- sonst wird hier ein Leerzeichen eingef�gt
    \hfill
    \begin{minipage}{0.5\textwidth}
        \centering
			\begin{align} 
				a_{z, handy} &= \tan(\alpha) \cdot {a_{y, handy}}\\
				a_y &= \frac{a_{y, handy}}{\cos(\alpha)}
			\end{align}
    \end{minipage}
\end{figure}

\begin{figure}[htb]
    \centering
    \begin{minipage}{0.5\textwidth}
        \centering
        \captionsetup{justification=centering}
          \begin{tikzpicture}
	\newcommand{\cosAngle}{0.866}
	\newcommand{\sinAngle}{0.5}
	\newcommand{\sideLength}{2}
	\newcommand{\handyLength}{4}
	\newcommand{\hlh}{\handyLength/2}
	\newcommand{\handyThickness}{0.25}
    	
	\coordinate (origo) at (0,0);
    	\coordinate (point_on_x) at (2,0);
	
	\begin{scope}[rotate=30]
		\coordinate (begin_a_z) at (-\hlh, 0);
	\end{scope}	

	\coordinate (end_a_y) at ($(begin_a_z)+(0,-\sideLength)$);
	\draw[thick,red,->] (begin_a_z) -- (end_a_y) node (y_real) [red,above left]
	{$a_z$}; 
	\begin{scope}[rotate=30]
		%\coordinate (end_a_z_handy) at ($(begin_a_z)+(0,-\sideLength)$);
		\coordinate (end_a_z_handy) at ($(begin_a_z)+(0,-\cosAngle*\sideLength)$);
	\end{scope}	
	\draw[thick,blue,->] (begin_a_z) -- (end_a_z_handy) node (z_handy) [blue,above
	right] {$a_{z, handy}$};

	\draw[thick,orange,->] (end_a_z_handy)  -- (end_a_y) node (y_handy)
	[orange,below right] {$a_{y, handy}$};

   	\begin{scope}[rotate=30]
		\coordinate (end_a_y_handy) at (-\hlh-\sideLength,0);
		%\coordinate (begin_a_z) at (-\hlh, 0);

    		\fill[gray] ($ (origo) + (-\hlh,-\handyThickness/2) $) rectangle ($ (origo) + (\hlh,\handyThickness/2) $);
		\draw[dashed, black] (-\hlh,0)--(\hlh,0);
		\coordinate (rotated) at (\hlh,0);
	\end{scope}	
    	% draw axes
    	\fill[black] (origo) circle (0.05);
    	\draw[thick,orange,->] (origo) -- ++(2,0) node[black,right] {$y$};
    	\draw[thick,blue,->] (origo) -- ++(0,-1.5) node (mary) [black,below] {$z$};
   	\draw[thick] ($ (origo) + (-1,0) $) -- ($ (origo) + (1,0) $);

	% angle axes
	\tkzMarkAngle[fill= orange,size=1.2cm,opacity=.4](point_on_x,origo,rotated)
    	\pic [draw, ->, "$\alpha$", angle eccentricity=1.5] {angle = point_on_x--origo--rotated};

	% angle vectors
	\tkzMarkAngle[fill=
	orange,size=1.2cm,opacity=.4](end_a_y,begin_a_z,end_a_z_handy) \pic [draw, ->,
	"$\alpha$", angle eccentricity=1.5] {angle = end_a_y--begin_a_z--end_a_z_handy};
  \end{tikzpicture}
        \caption{$a_z$ depending on Pitch}
    \end{minipage}% <- sonst wird hier ein Leerzeichen eingef�gt
    \hfill
    \begin{minipage}{0.5\textwidth}
        \centering
			\begin{align} 
				a_{y, handy} &= \tan(\alpha) \cdot a_{z, handy}\\ 
				a_z &= \frac{a_{z, handy}}{\cos(\alpha)}
			\end{align}
    \end{minipage}
\end{figure}

\FloatBarrier
\subsubsection{Acceleration depending on Roll}
\begin{figure}[htb]
    \centering
    \begin{minipage}{0.5\textwidth}
        \centering
        \captionsetup{justification=centering}
          \begin{tikzpicture}
	\newcommand{\cosAngle}{0.866}
	\newcommand{\sinAngle}{0.5}
	\newcommand{\sideLength}{2}
	\newcommand{\handyLength}{4}
	\newcommand{\hlh}{\handyLength/2}
	\newcommand{\handyThickness}{0.75}
    	
	\coordinate (origo) at (0,0);
    	\coordinate (point_on_x) at (2,0);

	\coordinate (end_a_y) at (-\hlh - \sideLength*\cosAngle,-1);
	

   	\begin{scope}[rotate=30]
		\coordinate (end_a_y_handy) at (-\hlh-\cosAngle*\sideLength,0);
		\coordinate (begin_a_y) at (-\hlh, 0);

    		\fill[gray] ($ (origo) + (-\hlh,-\handyThickness/2) $) rectangle ($ (origo) + (\hlh,\handyThickness/2) $);
		\draw[dashed, black] (-\hlh,0)--(\hlh,0);
		\coordinate (rotated) at (\hlh,0);
		 \draw[thick,green,->] (begin_a_y)  -- (end_a_y_handy) node (y_handy)
		 [green,below] {$a_{x, handy}$}; \begin{scope}[rotate=-30]
			\draw[very thick,red,->] (-\hlh*\cosAngle,-1)  -- (end_a_y) node (y_real)
			[red,above] {$a_x$}; 
			\draw[thick,blue,->](end_a_y_handy)
						--
						(-\hlh - \sideLength*\cosAngle,-1) node (z_handy) [blue, left] {$a_{z,
						handy}$};
		\end{scope}
	\end{scope}	
    	% draw axes
    	\fill[black] (origo) circle (0.05);
    	\draw[thick,green,->] (origo) -- ++(2,0) node[black,right] {$x$};
    	\draw[thick,blue,->] (origo) -- ++(0,-2) node (mary) [black,below] {$z$};
   	\draw[thick] ($ (origo) + (-1,0) $) -- ($ (origo) + (1,0) $);

	% angle axes
	\tkzMarkAngle[fill= green,size=1.2cm,opacity=.4](point_on_x,origo,rotated)
    	\pic [draw, ->, "$\beta$", angle eccentricity=1.5] {angle =
    	point_on_x--origo--rotated};

	% angle vectors
	\tkzMarkAngle[fill=
	green,size=1.2cm,opacity=.4](end_a_y,begin_a_y,end_a_y_handy) \pic [draw, ->,
	"$\beta$", angle eccentricity=1.5] {angle = end_a_y--begin_a_y--end_a_y_handy};
  \end{tikzpicture}
        \caption{$a_x$ depending on Roll}
    \end{minipage}% <- sonst wird hier ein Leerzeichen eingef�gt
    \hfill
    \begin{minipage}{0.5\textwidth}
        \centering
			\begin{align} 
				a_{x, handy} &= \tan(\beta) \cdot {a_{x, handy}}\\
				a_x &= \frac{a_{x, handy}}{\cos(\beta)}
			\end{align}
    \end{minipage}
\end{figure}

\begin{figure}[htb]
    \centering
    \begin{minipage}{0.5\textwidth}
        \centering
        \captionsetup{justification=centering}
          \begin{tikzpicture}
	\newcommand{\cosAngle}{0.866}
	\newcommand{\sinAngle}{0.5}
	\newcommand{\sideLength}{2}
	\newcommand{\mobileLength}{4}
	\newcommand{\hlh}{\mobileLength/2}
	\newcommand{\mobileThickness}{0.75}
    	
	\coordinate (origo) at (0,0);
    	\coordinate (point_on_x) at (2,0);
	
	\begin{scope}[rotate=30]
		\coordinate (begin_a_z) at (-\hlh, 0);
	\end{scope}	

	\coordinate (end_a_y) at ($(begin_a_z)+(0,-\sideLength)$);
	\draw[thick,red,->] (begin_a_z) -- (end_a_y) node (y_real) [red,above left]
	{$a_z$}; 
	\begin{scope}[rotate=30]
		%\coordinate (end_a_z_mobile) at ($(begin_a_z)+(0,-\sideLength)$);
		\coordinate (end_a_z_mobile) at ($(begin_a_z)+(0,-\cosAngle*\sideLength)$);
	\end{scope}	
	\draw[thick,blue,->] (begin_a_z) -- (end_a_z_mobile) node (z_mobile) [blue,above
	right] {$a_{z, mobile}$};

	\draw[thick,green,->] (end_a_z_mobile)  -- (end_a_y) node (y_mobile)
	[green,below right] {$a_{x, mobile}$};

   	\begin{scope}[rotate=30]
		\coordinate (end_a_y_mobile) at (-\hlh-\sideLength,0);
		%\coordinate (begin_a_z) at (-\hlh, 0);

    		\fill[gray] ($ (origo) + (-\hlh,-\mobileThickness/2) $) rectangle ($ (origo) + (\hlh,\mobileThickness/2) $);
		\draw[dashed, black] (-\hlh,0)--(\hlh,0);
		\coordinate (rotated) at (\hlh,0);
	\end{scope}	
    	% draw axes
    	\fill[black] (origo) circle (0.05);
    	\draw[thick,green,->] (origo) -- ++(2,0) node[black,right] {$x$};
    	\draw[thick,blue,->] (origo) -- ++(0,-1.5) node (mary) [black,below] {$z$};
   	\draw[thick] ($ (origo) + (-1,0) $) -- ($ (origo) + (1,0) $);

	% angle axes
	\tkzMarkAngle[fill= green,size=1.2cm,opacity=.4](point_on_x,origo,rotated)
    	\pic [draw, ->, "$\beta$", angle eccentricity=1.5] {angle = point_on_x--origo--rotated};

	% angle vectors
	\tkzMarkAngle[fill= green,size=1.2cm,opacity=.4](end_a_y,begin_a_z,end_a_z_mobile)
    	\pic [draw, ->, "$\beta$", angle eccentricity=1.5] {angle = end_a_y--begin_a_z--end_a_z_mobile};
  \end{tikzpicture}

        \caption{$a_z$ depending on Roll}
    \end{minipage}% <- sonst wird hier ein Leerzeichen eingef�gt
    \hfill
    \begin{minipage}{0.5\textwidth}
        \centering
			\begin{align} 
				a_{x, handy} &= \tan(\beta) \cdot a_{z, handy}\\ 
				a_z &= \frac{a_{z, handy}}{\cos(\alpha)}
			\end{align}
    \end{minipage}
\end{figure}

\FloatBarrier
\subsubsection{Acceleration depending on Azimuth}

\begin{figure}[htb]
    \centering
    \begin{minipage}{0.5\textwidth}
        \centering
        \captionsetup{justification=centering}
          \begin{tikzpicture}
	\newcommand{\cosAngle}{0.866}
	\newcommand{\sinAngle}{0.5}
	\newcommand{\sideLength}{2}
	\newcommand{\handyLength}{4}
	\newcommand{\hlh}{\handyLength/2}
	\newcommand{\handyWidth}{2}
	\newcommand{\hwh}{\handyWidth/2}
	\newcommand{\padding}{0.25}

	\begin{scope}[rotate=120]
		\draw[rounded corners] (-\hlh,-\hwh) rectangle (\hlh,\hwh);
		\draw[rounded corners] (-\hlh+\padding,-\hwh+\padding) rectangle (\hlh-\padding,\hwh-\padding);
		\draw[dashed, black] (-\hlh,0)--(\hlh,0);
		\draw[dashed, black] (0,-\hwh)--(0,\hwh);
		\coordinate (begin_a_x) at (0, -\hwh);
	\end{scope}
	
	\coordinate (end_a_x) at ($(begin_a_x)+(\sideLength,0)$);
	\draw[very thick,red,->] (begin_a_x)  -- (end_a_x) node (x_real) [red,below] {$a_x$}; 

	\coordinate (end_a_x_handy) at ($(begin_a_x)+(\cosAngle*\cosAngle*\sideLength,\sinAngle*\cosAngle*\sideLength)$);
	\draw[thick,green,->] (begin_a_x)  -- (end_a_x_handy) node (x_handy) [green,above] {$a_{x, handy}$};
	
	\draw[thick,orange,->] (end_a_x_handy) -- (end_a_x) node (y_handy) [orange,above right] {$a_{y, handy}$};

	\begin{scope}[rotate=30]
		\coordinate (rotated) at (0,2);
	\end{scope}
    	
	\coordinate (origo) at (0,0);
    	\coordinate (point_on_z) at (0,2);

	\coordinate (end_a_y) at (-\hlh - \sideLength*\cosAngle,-1);

    	% draw axes
    	\fill[black] (origo) circle (0.05);
    	\draw[thick,green,->] (origo) -- ++(2,0) node[black,right] {$x$};
    	\draw[thick,orange,->] (origo) -- ++(0,3) node (mary) [black,below left] {$y$};
   	\draw[thick] ($ (origo) + (-1,0) $) -- ($ (origo) + (1,0) $);

	% angle axes
	\tkzMarkAngle[fill= blue,size=1.2cm,opacity=.4](point_on_z,origo,rotated)
    	\pic [draw, ->, "$\gamma$", angle eccentricity=1.5] {angle =
    	point_on_z--origo--rotated};

	% angle vectors
	\tkzMarkAngle[fill=
	blue,size=1.2cm,opacity=.4](end_a_x,begin_a_x,end_a_x_handy) \pic [draw, ->,
	"$\gamma$", angle eccentricity=1.5] {angle = end_a_x--begin_a_x--end_a_x_handy};
  \end{tikzpicture}
        \caption{$a_x$ depending on Azimuth}
    \end{minipage}% <- sonst wird hier ein Leerzeichen eingef�gt
    \hfill
    \begin{minipage}{0.5\textwidth}
        \centering
			\begin{align} 
				a_{y, handy} &= \tan(\gamma) \cdot {a_{x, handy}}\\
				a_x &= \frac{a_{x, handy}}{\cos(\gamma)}
			\end{align}
    \end{minipage}
\end{figure}

\begin{figure}[htb]
    \centering
    \begin{minipage}{0.5\textwidth}
        \centering
        \captionsetup{justification=centering}
          \begin{tikzpicture}
	\newcommand{\cosAngle}{0.866}
	\newcommand{\sinAngle}{0.5}
	\newcommand{\sideLength}{2}
	\newcommand{\mobileLength}{4}
	\newcommand{\hlh}{\mobileLength/2}
	\newcommand{\mobileWidth}{2}
	\newcommand{\hwh}{\mobileWidth/2}
	\newcommand{\padding}{0.25}

	\begin{scope}[rotate=120]
		\draw[rounded corners] (-\hlh,-\hwh) rectangle (\hlh,\hwh);
		\draw[rounded corners] (-\hlh+\padding,-\hwh+\padding) rectangle (\hlh-\padding,\hwh-\padding);
		\draw[dashed, black] (-\hlh,0)--(\hlh,0);
		\draw[dashed, black] (0,-\hwh)--(0,\hwh);
		\coordinate (begin_a_y) at (\hlh, 0);
	\end{scope}
	
	\coordinate (end_a_y) at ($(begin_a_y)+(0, \sideLength)$);
	\draw[very thick,red,->] (begin_a_y)  -- (end_a_y) node (x_real) [red,below right] {$a_y$}; 

	\coordinate (end_a_y_mobile) at ($(begin_a_y)+(-\sinAngle*\cosAngle*\sideLength,\cosAngle*\cosAngle*\sideLength)$);
	\draw[thick,orange,->] (begin_a_y)  -- (end_a_y_mobile) node (y_mobile) [orange,above left] {$a_{y, mobile}$};
	
	\draw[thick,green,->] (end_a_y_mobile) -- (end_a_y) node (y_mobile) [green,above] {$a_{x, mobile}$};

	\begin{scope}[rotate=30]
		\coordinate (rotated) at (0,2);
	\end{scope}
    	
	\coordinate (origo) at (0,0);
    	\coordinate (point_on_z) at (0,2);

    	% draw axes
    	\fill[black] (origo) circle (0.05);
    	\draw[thick,green,->] (origo) -- ++(2,0) node[black,right] {$x$};
    	\draw[thick,orange,->] (origo) -- ++(0,3) node (mary) [black,below left] {$y$};
   	\draw[thick] ($ (origo) + (-1,0) $) -- ($ (origo) + (1,0) $);

	% angle axes
	\tkzMarkAngle[fill= blue,size=1.2cm,opacity=.4](point_on_z,origo,rotated)
    	\pic [draw, ->, "$\gamma$", angle eccentricity=1.5] {angle =
    	point_on_z--origo--rotated};

	% angle vectors
	\tkzMarkAngle[fill=	blue,size=1.2cm,opacity=.4](end_a_y,begin_a_y,end_a_y_mobile) 
	\pic [draw, ->,"$\gamma$", angle eccentricity=1.5] {angle = end_a_y--begin_a_y--end_a_y_mobile};
  \end{tikzpicture}
        \caption{$a_y$ depending on Azimuth}
    \end{minipage}% <- sonst wird hier ein Leerzeichen eingef�gt
    \hfill
    \begin{minipage}{0.5\textwidth}
        \centering
			\begin{align} 
				a_{x, handy} &= \tan(\gamma) \cdot a_{y, handy}\\ 
				a_y &= \frac{a_{y, handy}}{\cos(\gamma)}
			\end{align}
    \end{minipage}
\end{figure}