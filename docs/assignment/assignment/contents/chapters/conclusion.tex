In this final section, the application that was developed in the course of this
workshop is presented and some additional information like the link to the
project's presentation on hackster.io are provided.

\subsection{User Manual}
The way the user can interact with the application to
ease its shopping experience will be described subsequently.

\subsubsection{Starting SmartCart}
After starting SmartCart, the user will see the starting screen that is shown in
figure \ref{fig:start}. There are already some items added to the initial
shopping list. A user wanting to add additional items, might just push the
plus-button.

After the user is done with modifying the shopping list, the shopping can
be started by pushing on ``Go shopping''. The screen that is depicted in figure
\ref{fig:initial} is then shown and the user might now start to switch and add
items to the cart by performing gestures with the smartphone.

\begin{figure}[h]
\captionsetup{justification=centering}
\begin{subfigure}{0.475\textwidth} 
\centering 
\includegraphics[height= 0.3\textheight]{res/usermanual/startApp.png}
\caption{Start}
\label{fig:start}
\end{subfigure} \hspace{0.05\textwidth}
\begin{subfigure}{0.475\textwidth}
\centering 
\includegraphics[height= 0.3\textheight]{res/usermanual/initialShoppinglist.png}
\caption{Initial Shoppinglist}
\label{fig:initial}
\end{subfigure}
\caption{Initial State after Starting the App}
\label{fig:initialState}
\end{figure}

\subsubsection{Adding Items to the Cart}
To add the item which is written in big letters on the top of the list to the
cart, the user has to perform a circular gesture in clockwise direction.
The gesture starts at the bottom of the circle (6 o'clock), moves along the
left-hand side (9 o'clock), the top (12 o'clock) and back to the starting point
(passing 3 o'clock). The checked item will be moved to the bottom of your list
and will be written in crossed out letters (see figure
\ref{fig:firstItemChecked}). The app also provides its user feedback, if a
circle could be recognized and the item could be added to the cart (see figure
\ref{fig:feedbackCircle}).

\begin{figure}[h]
\captionsetup{justification=centering}
\begin{subfigure}{0.475\textwidth} 
\centering 
\includegraphics[height= 0.3\textheight]{res/usermanual/firstItemChecked.png}
\caption{First Item Checked Off}
\label{fig:firstItemChecked}
\end{subfigure} \hspace{0.05\textwidth}
\begin{subfigure}{0.475\textwidth}
\centering 
\includegraphics[height= 0.3\textheight]{res/usermanual/circleRecognized.png}
\caption{Feedback Circle Recognized}
\label{fig:feedbackCircle}
\end{subfigure}
\caption{Check Off Items}
\label{fig:checkItems}
\end{figure}

\subsubsection{Switching Items on the List}
If a user wants to add an item to the cart that is not on the top of the list,
he is able to switch the list by performing a circular gesture in the
counter-clockwise direction. The gesture is again started on the bottom of the
circle. Analogous to the case of adding an item to the cart, the SmartCart
application provides feedback to its user about the successful switching of the
list (see figures \ref{fig:beforeSwitching} and \ref{fig:afterSwitching}).

\begin{figure}[h]
\captionsetup{justification=centering}
\begin{subfigure}{0.475\textwidth}
\centering 
\includegraphics[height= 0.3\textheight]{res/usermanual/notswitched.png}
\caption{Shoppinglist before Switching}
\label{fig:beforeSwitching}
\end{subfigure} \hspace{0.05\textwidth}
\begin{subfigure}{0.475\textwidth}
\centering
\includegraphics[height= 0.3\textheight]{res/usermanual/switched.png}
\caption{Shoppinglist after Switching}
\label{fig:afterSwitching}
\end{subfigure}
\caption{Switch Items}
\label{fig:checkItems}
\end{figure}

%\FloatBarrier
\subsection{Additional Information}
The project SmartCart is published to hackster.io including a video that
presents the working gesture recognition independent of the smartphone's
orientation. The links to the publications as well as the
user credentials are listed in table \ref{tab:linksCredentials}. The other
resources that were offered in the workshop remained unchanged and are therefore
not listed in the table.

\begin{table}
\centering
\captionsetup{justification=centering}
\footnotesize
\begin{tabular}{p{0.2\textwidth}p{0.35\textwidth}p{0.35\textwidth}}
Resource & Credentials & Link \\
\hline
hackster.io & User: dsce.team.b@gmail.com \newline 
Pass: TWASoEQL &
\url{https://www.hackster.io/dcse-team-b/smart-cart-09155f} \\
youtube.com & User: dsce.team.b@gmail.com \newline 
Pass: TWASoEQL &
\url{https://www.youtube.com/channel/UCpFJqv0PWW_oIS3bQz22KmA} \\
\end{tabular}
\caption{Links and Credentials to/for additional Resources}
\label{tab:linksCredentials}
\end{table}